\documentclass[11pt, a4paper, reqno]{scrartcl}

\usepackage[utf8]{inputenc}
\usepackage{a4wide}
\usepackage{libertine}
\usepackage{graphicx}
\usepackage{listings}
\usepackage{xcolor}
\usepackage{float}
\usepackage{amsmath}
\usepackage{microtype}
\usepackage{hyperref}
\usepackage{pdflscape}

% for latex output of pandas
\usepackage{booktabs}

\begin{document}
    \title{Exercise No. 11}
    \author{David Bubeck, Pascal Becht, Patrick Nisbl\`e}
    \maketitle

    \lstset{
        language=Python,
        backgroundcolor=\color{gray!5},
        numbers=left,
        captionpos=t,
        breaklines=true,
        frame=l,
        xleftmargin=\parindent,
        basicstyle=\footnotesize\sffamily,
        keywordstyle=\bfseries\color{green!40!black},
        commentstyle=\itshape\color{purple!40!black},
        identifierstyle=\color{blue!60!black},
        stringstyle=\color{orange}
    }

    \section*{2 - Importance Sampling}
   
    	We should integrate the given integral with the Monte Carlo method.
    	\begin{align}
    		I = \frac{1}{\pi} \int_{-\infty}^{\infty} exp(-y_1^2 -y_2^2) \; dy_1 \; 			dy_2
    	\end{align}
    	
    	Now, due to important sampling, we can make an approximation of our 				integral as a sum and make a Monte Carlo estimation of the expected values. 		We obtain for our integral:
    	\begin{align*}
    		I = \frac{1}{\pi} \frac{1}{n} \sum_{i = 0}^{n} f(y_1, y_2)
    	\end{align*}
    	Where $n$ is the sample random number. \newline
    	Also we define a function $g(y_1, y_2) = \frac{f(y_1, y_2)}{p(y_1, y_2)}$ 			for the importance sampling, where $p(y_1, y_2)$ is a probability 					distribution function. Now we can check if our random values are in the 			given intervall. If it is the case the function $g(y_1, y_2)$ will return 			$1$ otherwise $-1$ to make a faster calculation. \newline
    	With the code we evaluated the integral for a value $I = 7367.86152808$ for 		a sampling number of $n = 1000$ in the given interval $[-5, 5]$.
     	
     	\begin{figure}[H]
     		\lstinputlisting[lastline=30]{exercise11_1.py}
     	\end{figure}
      
		        
\end{document}