\documentclass[11pt, a4paper, reqno]{scrartcl}

\usepackage[utf8]{inputenc}
\usepackage{a4wide}
\usepackage{libertine}
\usepackage{graphicx}
\usepackage{listings}
\usepackage{xcolor}
\usepackage{float}
\usepackage{amsmath}
\usepackage{microtype}

% for latex output of pandas
\usepackage{booktabs}

\begin{document}
    \title{Exercise No. 8}
    \author{David Bubeck, Pascal Becht, Patrick Nisbl\`e}
    \maketitle
    
    \lstset{
        language=Python,
        backgroundcolor=\color{gray!5},
        numbers=left,
        captionpos=t,
        breaklines=true,
        frame=l,
        xleftmargin=\parindent,
        basicstyle=\footnotesize\sffamily,
        keywordstyle=\bfseries\color{green!40!black},
        commentstyle=\itshape\color{purple!40!black},
        identifierstyle=\color{blue!60!black},
        stringstyle=\color{orange}
    }

    \section{Stability Analysis}
    
    \begin{align}
        \frac{\text d N_i}{\text d t} =& N_i\left( a_i-N_i-\sum_j b_{i,j}P_j \right) \\
        \frac{\text d P_i}{\text d t} =& P_i\left( \sum_j c_{i,j}N_j -d_i \right)
    \end{align}
    \subsection{fixed Points}
    to find the two fixpoints we can guess the trivial and test it
    \begin{align}
        FP_1:P=\begin{pmatrix}
            0\\0\\0
        \end{pmatrix}, N=\begin{pmatrix}
            0\\0\\0
        \end{pmatrix}
    \end{align}
    which gives us a solution for $\frac{\text d N_i}{\text d t} = 0$ and $\frac{\text d P_i}{\text d t} = 0$
    
    the second fixed point we guess the following
    \begin{align}
        FP_2:P=\begin{pmatrix}
        1\\1\\1
        \end{pmatrix}, N=\begin{pmatrix}
        1\\1\\1
        \end{pmatrix}
    \end{align}
    
    and with these values we also get $\frac{\text d N_i}{\text d t} = 0$ and $\frac{\text d P_i}{\text d t} = 0$
    
    \subsection{Jacobi Matrix A}
    
    \begin{align}
        A =& \left(\begin{smallmatrix}
            a_1-2N_1-\sum_j b_{1,j}P_j & 0 & 0 & N_1b_{1,1} & N_1b_{1,2} & N_1b_{1,3}\\
            0 & a_2-2N_2-\sum_j b_{2,j}P_j & 0 & N_2b_{2,1} & N_2b_{2,2} & N_2b_{2,3}\\
            0 & 0 & a_3-2N_3-\sum_j b_{3,j}P_j & N_3b_{3,1} & N_3b_{3,2} & N_3b_{3,3}\\
            P_1c_{1,1} & P_1c_{1,2} & P_1c_{1,3} & \sum_j c_{1,j}N_j -d_1 & 0 & 0\\
            P_2c_{2,1} & P_2c_{2,2} & P_2c_{2,3} & 0 & \sum_j c_{2,j}N_j -d_2 & 0\\
            P_3c_{3,1} & P_3c_{3,2} & P_3c_{3,3} & 0 & 0 & \sum_j c_{3,j}N_j -d_3\\
        \end{smallmatrix}\right)\\
        \overset{FP_2}{=}& \begin{pmatrix}
            -1 & 0 & 0 & 20 & 30 & 5\\
            0 & -1 & 0 & 1 & 3 & 7\\
            0 & 0 & -1 & 4 & 10 & 20\\
            20 & 30 & 35 & 0 & 0 & 0\\
            3 & 3 & 3 & 0 & 0 & 0\\
            7 & 8 & 20 & 0 & 0 & 0\\
        \end{pmatrix}
    \end{align}
    
    \subsection{EVals and EVecs}
        \begin{figure}[H]
            \centering
            \begin{tabular}{c|r}
                i & $\lambda_i$\\
                \hline
                1 & -33.06613885\\
                2 & 32.06613885\\
                3 & -9.99024416\\
                4 &  8.99024416\\
                5 & -0.86313363\\
                6 & -0.13686637\\
            \end{tabular}
        \end{figure}
        and 
        \begin{align}v_i = 
            \begin{pmatrix}
                0.5562403 & -0.5463627 & 0.82241683 & -0.80651197 & 0.48435472 & 0.09294963\\
                0.10307497 & -0.10124458 & -0.1459207 & 0.14309872 & -0.29627361 & -0.0568561\\
                0.32248203 & -0.31675546 & -0.36432112 & 0.35727545 & -0.0551648 & -0.01058635\\
                -0.67377404 & -0.68244818 & -0.29655807 & -0.3231716 & 0.67224106 & 0.81356161\\
                -0.08907577 & -0.09022253 & -0.09374396 & -0.10215667 & -0.46197821 & -0.55909667\\
                -0.33774498 & -0.3420931 & 0.26995038 & 0.2941761 & 0.09616338 & 0.11637914\\
            \end{pmatrix}
        \end{align}
        where every column is the corresponding Eigenvector
\end{document}