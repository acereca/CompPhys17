\documentclass[11pt, a4paper, reqno]{scrartcl}

\usepackage[utf8]{inputenc}
\usepackage{a4wide}
\usepackage{libertine}
\usepackage{graphicx}
\usepackage{listings}
\usepackage{xcolor}
\usepackage{float}
\usepackage{amsmath}

% for latex output of pandas
\usepackage{booktabs}

\begin{document}
    \title{Exercise No. 5}
    \author{David Bubeck, Pascal Becht, Patrick Nisbl\`e}
    \maketitle
    
    \lstset{
        language=Python,
        backgroundcolor=\color{gray!10},
        numbers=left,
        captionpos=t,
        breaklines=true,
        frame=l,
        xleftmargin=\parindent,
        basicstyle=\footnotesize\sffamily,
        keywordstyle=\bfseries\color{green!40!black},
        commentstyle=\itshape\color{purple!40!black},
        identifierstyle=\color{blue!60!black},
        stringstyle=\color{orange}
    }

    \newpage
    \section*{2 - Tridiagonal matrices}

    	\subsection*{1)}
			Due to the high amount of zeros in the matrix the Gaußian elimination 				can be simplifyed to
			
			\begin{align}
				\mbox{for \, $i = 2 ... n$}
				\begin{cases}
					a_i = \frac{a_i}{b_{i-1}} \\
					b_i = b_i - a_i \cdot c_{i-1} \\
					r_i = r_i - a_i \cdot r_{i-1}
				\end{cases}
			\end{align}
			
			However, $b_1$ and $c_1$ will not be changed.
    		

		\subsection*{2)}
			Now we can use the new found values of $b_i$ and $r_i$ to do a backward 			substitution. We start the process with the last row, by which the 					entry of the solution vector is 
			
			\begin{align}
				x_n = \frac{r_n}{b_n}
			\end{align}
						
			For the other entries we obtain with a recursion formula
			
			\begin{align}
				x_i = \frac{r_i - c_i \cdot x_{i+1}}{b_i}
			\end{align}			    		
    	
    	\subsection*{3)}
    		To write the code we use the formula above and implement it as a 					numerical subroutine.
    		
    		\begin{figure}[H]
        		\lstinputlisting[language=Python,
        			caption={Exercise05.py},
        			lastline=21]{Exercise05.py}        
    		\end{figure}
    	
    	\subsection*{4)}
    		To test the algorithm we set a matrix with the values to all $a = -1$, 				all $b = 2$, all $c = -1$ and all $r = 0.1$.
    		
    		\begin{figure}[H]
        		\lstinputlisting[language=Python,
        			caption={Exercise05.py}, firstline=24, firstnumber=24,
        			lastline=44]{Exercise05.py}        
    		\end{figure}
    		
    		For the vector $x$ we obtain
    		
    		\begin{align*}
    			x =
    			\begin{pmatrix}
    				0.5 \\
    				0.9 \\
    				1.2 \\
    				1.4 \\
    				1.5 \\
    				1.5 \\
    				1.4 \\
    				1.2 \\
    				0.9 \\
    				0.5 
    			\end{pmatrix}
    		\end{align*}
  
    	\subsection{5)}
    		We put the solution back into the original matrix equation to check if 				the algorithm works correctly. Therefore we use the equations below for 			a numerical subroutine.
    		
    		\begin{align}
    			a_1 \cdot x_n + b_1 \cdot x_1 + c_1 \cdot x_2 = r_1, \\
					i = 2, ..., n-1 \;\;\;\;     			
    			a_i \cdot x_{i-1} + b_i \cdot x_i + c_i \cdot x_{i+1} = r_i,\\
    			a_n \cdot x_{n-1} + b_n \cdot x_n + c_n \cdot x_1 = r_n
    		\end{align}
    		
    		\begin{figure}[H]
        		\lstinputlisting[language=Python,
        			caption={Exercise05.py}, firstline=46, firstnumber=46,
        			lastline=53]{Exercise05.py}        
    		\end{figure}
  
  			For the vector $r$ we obtain
    		
    		\begin{align*}
    			R =
    			\begin{pmatrix}
    				0.1 \\
    				0.1 \\
    				0.1 \\
    				0.1 \\
    				0.1 \\
    				0.1 \\
    				0.1 \\
    				0.1 \\
    				0.1 \\
    				0.1 
    			\end{pmatrix}
    		\end{align*}
  			
  			We get the same solution we put into our algorithm, $R = r$, therefore 				is no deviation visible.
    		
    	

  
    	
    		

\end{document}